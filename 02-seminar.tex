\chapter{SAT encoding of cryptographic problems}

\tdi{citacie novych clankov (n-ary xor, ...)}

\begin{framed}
\noindent Veci, ktoré už boli v predošlej verzií (na preddiplomový projekt) som sem nedal. Výsledný text asi využije časti z oboch ale väčšina bude nová resp. prepísaná.
\end{framed}

\tdi{Intro}

\section{Typical problems}
\subsection{Preimage attack on hashfunctions}

\section{Required operations}
\tdi{xor/and/or/..., add, shift/rotate, ...}

\chapter{Library design}

A result of this work is a software library for the Python programming language designed to automate and simplify the generation of SAT instances for various cryptographic problems.
In this section we describe the internal design and working of this library with examples of usage.

\section{Related work}

\begin{framed}
\noindent Tu niečo o tom, že toto už niekto robil. Zdrojáky nie sú dostupné, ale možno by bolo fajn ich získať od autorov, nech je tento popis existujúcich riešení aspoň trochu zaujímavý.
\end{framed}

\section{Creating boolean circuits}

To create a SAT instance for some cryptographic problem, such as pre-image attacks on hash functions, we can proceed in the following way:
We use a special data type from our library to represent binary vectors (words).
These can be of any length to accommodate algorithms working with different word sizes.
Some or all bits can be set to a constant value, others will become variables and their value will be subject to computation.

Next, we will implement the desired cryptographic algorithm, using these special data types instead of the built-in data types representing machine words.
The operations performed will be stored in an acyclic directed graph (DAG) which will represent the boolean circuit for this algorithm.

With this circuit we can do further modifications, such as setting some free output variables to constant values.
Afterwards this can be exported in a format suitable for a SAT solver (for example by using the \emph{Tseitin} transformation to create a CNF instance).
After the instance is solved by a SAT solver (or proved unsatisfiable) the library will read back the valuation of all variables for further querying by the library user.

\section{Operator overloading}

The core idea of this library is to make us of a feature present in many popular programming languages called \emph{operator overloading}.
In simple terms, this allows library designers to specify custom behavior of their types when applied to various operators.

A typical example would be a custom data type designed to represent complex numbers.
In languages without operator overloading such as \emph{C} one would have to use special helper functions to perform operations on these numbers, writing code like \texttt{C = complex\_multiply(A, B)}.
This is not only verbose and cumbersome but in case of more complicated expressions it hides the intent from the reader.

Operator overloading allows us to define what should happen when the built in multiplication operator of the language is applied to our new data type, resulting in clean and readable code such as \texttt{C = A*B}.

~ \\

In the case of our library the use of operator overloading also provides another important benefit.
Since the code working with the binary vector data type is identical to code working with built-in types for machine words we can reuse existing code.

For example, to create a boolean circuit for some cryptographic hash function, the library user doesn't have to implement the complete algorithm from scratch.
It is enough to take a existing, traditional implementation and modify it slightly -- the biggest (and in most cases only) change being that the initial types and values of all variables and constants are changed to the new binary vector data type.
This makes it much easier and much less error prone to create a boolean circuit for some cryptographic algorithm.

\section{Internal representation}

Performing any operation on one or more binary vectors will result in a new binary vector, which remembers its operands and what operation (such as boolean and, xor, modular addition or cyclic shift) was performed.
This results in an acyclic directed graph of the computation, since some initial constants or intermediate results can be used in multiple operations, but no result can depend (directly or transitively) on itself.

Depending on the type of the binary vector (constant or free input) and the operation performed, different classes of objects are used internally.
However all of them are subtypes of the binary vector type and thanks to polymorphism they can be used interchangeably. 

\section{Optimizations}

Using the described methods to build the boolean circuit for some cryptographic algorithm is simple for the library user, however in some cases the result is far from optimal or efficient.
This is due to the fact that the circuit is built step by step without any forethought and with every operation only new nodes are added to it.

To overcome this problem and to improve the resulting SAT instance we can perform various optimizations on this boolean circuit before it is converted and exported.

\subsection{Merging binary operators}

One possible suboptimal behavior results from the way binary operators in Python (and most other popular programming languages) behave.
Most operators are left associative, meaning that they are evaluated from the left to right.
For example, given an expression \texttt{A \& B \& C \& D} it will be evaluated as \texttt{((A \& B) \& C) \& D}, where the \texttt{\&} operator represents boolean and.

Since we are overloading these operators the same behavior applies to the creation of the boolean circuit from binary vectors.
If we denote \texttt{N(x,y)} the result of overloaded operator \texttt{x \& y}, the above expression would result into \texttt{N(N(N(A, B), C), D)}.

That means combining four inputs would produce three nodes in the boolean circuit, each with two inputs.
However, we could easily create a version of the \emph{and} node with arbitrary arity and replace this with a single node \texttt{N(A, B, C, D)}.

This not only reduces the size of the boolean circuit but also of the resulting instance. 
The Tseitin transformation requires one variable for each node and $d+1$ clauses for $d$-ary \emph{AND} or \emph{OR} predicates.
Given an expression like $\texttt{A}_\texttt{1} ~\&~ \cdots ~\&~ \texttt{A}_\texttt{n}$ this optimization reduced number of intermediate variables from $n-1$ to $1$ and the number of clauses from $3n-3$ to $n+1$.

This optimization is implemented by traversing the graph in topological order from inputs to outputs.
Each time a node representing either \emph{AND} or \emph{OR} operators is processed the operands are checked.
If any of the operands is of the same type, its operands are added as operands of the current node and the operand node is discarded.
This way the number of nodes in the graph is reduced while the arity of the current operand is increased.

\subsection{Merging XOR operators}
Another common binary operator used in cryptographic constructions is the exclusive or.
Unlike \emph{AND} and \emph{OR} however it isn't desirable to merge multiple \emph{XOR} operators together and increase the arity.
The reason is that when using the Tseitin transformation to convert $d$-ary exclusive or to clauses in the conjunctive normal form we need $2^d$ clauses.
Since this number grows exponentially the linear saving in number or intermediate variables is not worth it in many cases.

Therefore our library provides an optimizer that merges \emph{XOR} operators only up to a specified maximum arity.
The choice of this value depends on many factors like the type of problem being solved or the SAT solver used.
\begin{framed}
\noindent Tu\citep{bard2007efficient} píšu, že 6 je optimálne. Čo som zatiaľ skúšal tak to bolo vždy pomaľšie ako binárne. Treba to vyskúšať ešte v kombinácii s ďaľšími optimalizáciami a možno na iných inštanciách alebo SAT solveroch.
\end{framed}

Another possibility is to output the \emph{XOR} expressions in special format for SAT solvers that support it (such as \emph{CryptoMiniSAT}).
In those cases we want to merge without any arity limits, same as in the case of \emph{AND} and \emph{OR} operators, since the resulting output will be linear, not exponential, in the arity.

\subsection{Other optimizations}
In this section we briefly describe other optimizations that can be performed on the boolean circuit.

\subsubsection{Eliminating pointless nodes}
Some kinds of nodes can be eliminated from the circuit completely by instead using a different variable mapping.
This is useful because each node in the circuit will need intermediate variables in the Tseitin transformation.
While these might be easy for the SAT solver to satisfy, the size of the instance grows needlessly.

For example, a node for cyclic rotation of the binary vector can be simply removed as long as the variables in all nodes which used this rotated vector are also shifted.
This can be achieved simply by storing, in addition to the operands for every operator node, also the mapping between the variables of this node and the variables of the particular operand.
The same principle can be applied to binary shifts with some minor modifications.

Another example is the binary negation operator, which can be eliminated if we also keep a flag in the variable mapping indicating if the variable should be negated or not.

Since shifts, rotations and negations are quite common in most cryptographic constructions using these optimizations helps reduce the size of the instance, both in number of clauses as well as variables.

\subsubsection{Constant propagation}
Many cryptographic primitives begin with several constants that are used throughout the computation.
In cases where part of the result can be computed, such as when performing the \emph{AND} operation with one operand being constant with some zero bits, computing this result before export can also serve to eliminate pointless variables and clauses.

\tdi{adders?}
\tdi{custom operators}

\section{SAT solvers}

Interfacing with the \emph{SAT} solver is done using 	text files.
Most available SAT solvers support input and output using text files in the \emph{DIMACS} format.

In this format, variables are nonzero numbers.
A negative number represents the negation of variable with the same absolute value.
Individual clauses in the conjunctive normal form are then provided as lines in the file, each line listing the variables of a particular disjunction.

After running a SAT solver on this file we obtain a satisfying valuation of the variables (if one exists).
This is again a simple text file, in which each variable number from input appears once, either positive or negative.

Some SAT solvers provide Python interfaces which make interacting with them simpler.
However, in our case this doesn't provide much benefit since the user of the library does not need to deal with the SAT solver interfacing.
Therefore we have decided against using these kinds of interfaces.

\subsection{Export}

Once we have a boolean circuit for some cryptographic algorithm (and after we have run optional optimizations) we need to export it in the format described above.

\subsubsection{Assigning variables}
The first step is to assign variable numbers to both all input and output variables as well as all the internal variables resulting from the Tseitin transformation.
To do this we simply keep a variable counter and traverse the circuit.
Each time we reach a node with unassigned variables we can use the current value of the counter as the new variable number and then increment the counter.

\subsubsection{Generating clauses}
After all variables have their numbers assigned we can proceed with generating the resulting clauses in conjunctive normal form.
Again we traverse the circuit and write those clauses to a file one node at a time.
Each binary vector node type (such as a constant, \emph{AND} operator, modular addition and so on) then prints the clauses that describe its value as a function of its operands.

\subsection{Running the SAT solver}
After the boolean circuit has been written to a file as a CNF instance, we can run any compatible SAT solver.
This is done simply as executing a user-provided binary and waiting for the result.

In case no satisfiable assignment of variables exists this is reported to the user.
Otherwise we proceed with the next step which is the loading of the assignment.

\subsection{Import}
To give the library user a way to extract the resulting data easily we read the satisfiable assignment of all variables as provided by the SAT solver and store it in memory as a mapping from variable number to the truth value.
The binary vector data type remembers the variable numbers for each of its bits.
Therefore when queried for value of any of its bits it can determine this from the mapping.

For example, in the case of pre-image attacks on hash functions this allows the user to read back the input for the hash function which hashes to the desired specified output.

\chapter{Measurements}
\tdi{priklady instancii, velkosti, cas, ...}
\tdi{compare to others}

...\\

\tdi{nove citacie pridat do bib}